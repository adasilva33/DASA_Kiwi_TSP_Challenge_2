\section{Introduction}

The number of flight connections continues to grow each year \cite{statista_flights_year}, with over 38 million flights scheduled in 2023. This increasing volume poses a significant challenge for travellers trying to find the best and cheapest flight connections for their specific journeys, particularly when multiple cities are involved. As a result, travel agencies have implemented online trip planner algorithms to help travellers find flights that meet their requirements. Examples of such platforms include Google Flights, OpenFlights.org, Skyscanner, Kayak, and Kiwi.com. 
%
These agencies have introduced various challenges to develop more powerful trip planner algorithms. For example, as noted in \cite{reinforcement_learning_yaro}, OpenFlights.org launched the Air Travelling Salesman Project. Similarly, in 2017, Kiwi.com initiated the Travelling Salesman Challenge, which led to the development of their current algorithm. In 2018, Kiwi.com introduced a new challenge, the Travelling Salesman Problem 2.0, which is the focus of this study. Despite the large number of participants in these challenges, there is limited literature on the methods employed. The winning team used a breadth-first search (BFS) algorithm \cite{tsp2_award}, while other participants applied heuristics such as simulated annealing, genetic algorithms, and reinforcement learning. Only two papers have been published on these approaches \cite{reinforcement_learning_yaro,local_search_yaro}, focusing on local search and reinforcement learning. This scarcity of research inspired our decision to implement a novel solution using Monte Carlo Tree Search (MCTS) to tackle the problem. 

The problem at hand is a variant of the well-known Travelling Salesman Problem (TSP). It can be described as a generalised, asymmetric, and time-dependent TSP. A traveller must visit a set of areas, one per day, starting from a given airport, with various flight connections available between these airports on different days. The objective is to determine the cheapest flight route that allows the traveller to return to the starting location. Due to the large number of possible routes, solving this problem by exhaustively exploring every potential solution is infeasible. Therefore, a heuristic approach is employed. 
